\documentclass[12pt]{article}
\usepackage{amsmath}
\title{Matte-Calculus}
\author{Eivind}
\begin{document}
%\Author{Eivind R.}
%%%%%%%%%%%% Potens- og logaritmeregler %%%%%%%%%%%%
\section{Potens- \\og logaritmeregler}

\[a^0 = 1\]
\[a^{-x}=\frac{1}{a^{x}}\]
\[a^{x^y}=a^{xy}\]
\[a^{x^y}=a^{xy}\]
\[a^{x+y}=a^{x}\cdot a^{y}\]
\[a^{x-y}=\frac{a^{x}}{a^{y}}\]
\[(ab)^{x}= a^{x} \cdot b^{x}\]
\[log_{a} 1 = 0\]
\[log_{a} (x^{y}) = y\cdot log_{a}x\]
\[log_{a} (x\cdot y) = log_{a} x + log_{b} y\]
\[log_{a} (\frac{x}{y}) = log_{a} x -log_{a} y\]
\[log_{a} x =\frac{log_{b} x}{log_{b} a}\]
\[log_{e} x = ln x\]

%%%%%%%%%%%% Trigonometri %%%%%%%%%%%%
\section{Trigonometri}
%%%%%%%%%%%%
\subsection*{Sin,Cos,Tan definisjon}
\[\sin \theta = \frac{a}{h}\]
\[\cos \theta = \frac{b}{h}\]
\[\tan \theta = \frac{a}{b} \quad eller \quad \frac{\sin \theta}{\cos \theta}\]
\[\csc \theta = \frac{h}{a} \quad eller \quad \frac{1}{\sin \theta}\]
\[\sec \theta = \frac{h}{b} \quad eller \quad \frac{1}{\cos \theta}\]
\[\cot \theta = \frac{b}{a} \quad eller \quad \frac{1}{\tan \theta}\]
%%%%%%%%%%%%
\subsection*{Trigonometriske \newline Identiteter}
\[sin (s\pm t) = sin(s) \cdot cos(t) \pm cos(s) \cdot sin(t) \]
\[cos (s\pm t) = cos(s) \cdot cos(t) \mp sin(s) \cdot sin(t) \]
\[sin^{2}(t) + cos^{2}(t) = 1\]
%%%%%%%%%%%%
\subsection*{Sinussetningen}
\[\frac{\sin A}{a}=\frac{\sin B}{b}=\frac{\sin C}{c}\]
eller
\[\frac{a}{\sin A}=\frac{b}{\sin B}=\frac{c}{\sin C}\]
%%%%%%%%%%%%
\subsection*{Cosinussetningen}
\[a^2=b^2+c^2-2bc\cos A\]
hvor a er motstaende side i trekanten.
%%%%%%%%%%%%
\subsection*{Arealsetningen}
v er vinkel mellom sidene b og c.
\[Areal=\frac{1}{2}bc\sin v\]
%%%%%%%%%%%%
\subsection*{Likning for en sirkel}
En sirkel med sentrum (h,k) og radius $a>0$
\[(x-h)^{2}+(y-k)^{2}=a^2\]
En sirkel sentrum i (0,0) har likning
\[x^{2}+y^{2}=a^2\]


%%%%%%%%%%%% Funksjoner %%%%%%%%%%%%
\newpage\section{Funksjoner}
%%%%%%%%%%%%
\subsection*{Intervaller}
Apent intervall
\[(a,b) \Leftrightarrow a<x<b\]
Lukket intervall
\[[a,b] \Leftrightarrow a<x<b\]
Halvapent intervall
\[[a,b) \Leftrightarrow a\leq x<b\]
\[(a,b] \Leftrightarrow a<x\leq b\]
%%%%%%%%%%%%
\subsection*{Definisjon - Definisjonsmengde og verdimengde}
Definisjonsmengde = D(f) = Domain = alle mulige x-verdier.
\newline Verdimengde = R(f) = Range = verdier funksjoner kan oppna.
%%%%%%%%%%%%
\subsection*{Definisjon - Like og oddefunksjoner}
En likefunksjon(even) 
\newline f(-x)=f(x) for hver x i definisjonsmenden til f.
\newline Grafen er symertrisk om y-aksen.
\newline En oddefunksjon
\newline f(-x)=-f(x) for hver x i definisjonsmenden til f
\newline Grafen er symetrisk om origo.
%%%%%%%%%%%%
\subsection*{Definisjon - one-to-one}
$f(x_1)\neq f(x_2) \Rightarrow x_1 \neq x_2$
\newline$f(x_1)=f(x_2) \Rightarrow x_1 = x_2$
\newline Krav: $x_1 og x_2$ ma tilhore verdimengden til f
%%%%%%%%%%%%
\subsection*{Definisjon Den inverse funksjonen}
$y=f^{-1}(x) \Leftrightarrow f(y)=x$
\newline Funksjoner som opphever" hverandre
\newline$f(f^{-1}(x))=x$
\newline$f(f^{-1}(y))=y$
\newline Krav: f ma være one-to-one"
%%%%%%%%%%%%
\subsection*{Egenskaper til inverse funksjoner}
Grafen til $f^{-1}$ er refleksjonen av grafen til f om linja x=y
\newline The domain of $f^{-1}$ is the range of f
\newline The range of $f^{-1}$ is the domain of f
%%%%%%%%%%%%
\subsection*{Definisjon Den begrensede sinus funksjonen}
\[Sin x=\sin x,hvis -\frac{\pi}{2}<x<\frac{\pi}{2}\]
%%%%%%%%%%%%
\subsection*{Definisjon Den begrensede tangens funksjonen}
\[Sin x=\sin x,hvis -\frac{\pi}{2}<x<\frac{\pi}{2}\]
%%%%%%%%%%%%
\subsection*{Definisjon Sinus invers}
\[y=\sin^{-1}x\Leftrightarrow x=\sin y,hvis -\frac{\pi}{2}\leq x\leq\frac{\pi}{2}\]
\[\sin^{-1}(\sin x)=\arcsin(\sin x)=x,for-\frac{\pi}{2}\leq x\leq\frac{\pi}{2}\]
\[\sin(sin^{-1})=\sin(\arcsin x)=x,for-1\leq x\leq1\]
%%%%%%%%%%%%
\subsection*{Definisjon Tangens invers}
\[y=\tan^{-1}x\Leftrightarrow x=\tan y,hvis -\frac{\pi}{2}<x<\frac{\pi}{2}\]
\[\tan^{-1}(\sin x)=\arctan(\sin x)=x,for-\frac{\pi}{2}<x<\frac{\pi}{2}\]
\[\tan(sin^{-1})=\tan(\arctan x)=x,for-\infty<x<\infty\]
%%%%%%%%%%%%
\subsection*{Hyperboliske funksjoner}
cosh x = $\frac{e^x + e^{-x}}{2}$
\newline sinh x = $\frac{e^x - e^{-x}}{2}$
\newline cosh x = $\frac{e^x - e^{-x}}{2}$
\newline tanh x = $\frac{sinh}{cosh}=\frac{e^x - e^{-x}}{e^x + e^{-x}}$
\newline sech x = $\frac{1}{cosh}=\frac{2}{e^x + e^{-x}}$
\newline csch x = $\frac{1}{sinh}=\frac{2}{e^x - e^{-x}}$
\newline coth x = $\frac{cosh}{sinh}=\frac{e^x + e^{-x}}{e^x - e^{-x}}$
%%%%%%%%%%%%
\subsection*{Funksjon-Invers funksjon}
\[y=e^{x}\Leftrightarrow x=ln(y), y>0\]
\[\]


%%%%%%%%%%%% Grenseverdier %%%%%%%%%%%%
\newpage\section{Grenseverdier}
%%%%%%%%%%%%
\subsection*{Definisjon Limit}
Vi sier at f(x) nærmer seg grenseverdien L mens x nærmer seg a, og vi skriver:
\newline$\lim_{x\to0} f(x) = L$
%%%%%%%%%%%%
\subsection*{Definisjon left og right limit}
x narmer seg a fra venstre
\newline$\lim_{x\to a-} f(x) = L$
\newline x narmer seg a fra hoyre
\newline$\lim_{x\to a-} f(x) = L$
\newline Sammenheng:
\newline$\lim_{x\to a} f(x)\iff\lim_{x\to a-} f(x) = \lim_{x\to a+} f(x)$
%%%%%%%%%%%%
\subsection*{The Squeeze Theorem}
$f(x)\leq g(x)\leq h(x)$
\newline$\lim_{x\to a} f(x) = \lim_{x\to a} h(x) = L$
\newline$\Rightarrow\lim_{x\to a} g(x) = L$
%%%%%%%%%%%%
\subsection*{Definisjon Continuity}
f er continuous på et punkt c hvis
\newline$\lim_{x\to c}f(x) = f(c)$
\newline Left continuous
\newline$\lim_{x\to c-}f(x) = f(c)$
\newline Right continuous
\newline$\lim_{x\to c+}f(x) = f(c)$
%%%%%%%%%%%%
\subsection*{Noen grenseverdier}
\[\lim_{x\to\infty}ln x=\infty \]
\[\lim_{x\to0+}ln x=-\infty\]
%%%%%%%%%%%%
\subsection*{L'Hopitals Regel}


%%%%%%%%%%%% Derivasjon %%%%%%%%%%%%
\newpage\section{Derivasjon}
%%%%%%%%%%%%
\subsection*{Ulik notasjon}
\[D_x y=y'=\frac{dy}{dx}=\frac{d}{dx}f(x)=f'(x)=D_x f(x)=Df(x)\]
%%%%%%%%%%%%
\subsection*{Tangentlinjer og deres stigningstall}
Hvis funksjonen er continuous pa $x = x_0$ og grenseverdien eksisterer.
\[m =\lim_{h\to0}\frac{f(x+h)-f(x)}{h}=\lim_{h\to0}\frac{\Delta y}{\Delta x}=\frac{dy}{dx}\]
Tangentlinja er gitt ved:
\[y = y_0 + m(x-x_0)\]
og vil passere gjennom punktet$P=(x_0,f(x_0))$
\newline hvor $f(x_0)=y_0$
%%%%%%%%%%%%
\subsection*{Definisjon - Gjennomsnittlig vekst fra x til (x+h)}
\[\frac{f(x+h)-f(x)}{h}=\frac{\Delta y}{\Delta x}\]
%%%%%%%%%%%%
\subsection*{Definisjon - Den deriverte(momentan vekst i x)}
Den deriverte til en funksjon er en annen funsjon definert.
\newline Punkter der funksjonen ikke er deriverbar kaller vi singulare punkter.
\[f'(x)=\lim_{h\to0}\frac{f(x+h)-f(x)}{h}=\lim_{h\to0}\frac{\Delta y}{\Delta x}=\frac{dy}{dx}\]
%%%%%%%%%%%%
\subsection*{Theorem - Sum,differanse og konstanter}
\[(f+g)'(x) = f'(x)+g'(x)\]
\[(f-g)'(x) = f'(x)-g'(x)\]
\[(Cf)'(x) = Cf'(x)\]
%%%%%%%%%%%%
\subsection*{Derivasjon av Sinus}
\[(\sin x)'=\cos x\]
For å bevise dette ma vi bevise at $\lim_{\to0}\frac{\sin x}{x}=1$ 
\subsection*{Derivasjon av Cosinus}
\[(\cos x)'=-\sin x\]
For å bevise dette ma vi bevise at $\lim_{\to0}\frac{\cos x -1}{x}=1$
%%%%%%%%%%%%
\subsection*{Derivasjon av Tangens}
\[\]
%%%%%%%%%%%%
\subsection*{Potensregelen}
\[(x^n)' = n \cdot x^{n-1}\]
%%%%%%%%%%%%
\subsection*{Produktregelen}
\[(u\cdot v)' = u' \cdot v + u\cdot v'\]
%%%%%%%%%%%%
\subsection*{Brokregelen}
Krav: g(x) $\neq$ 0
\[(\frac{f}{g})'(x) = \frac{f'(x)\cdot g(x) - g'(x) \cdot f(x)}{(g(x))^2}\]
%%%%%%%%%%%%
\subsection*{Kjerneregelen}
\[\frac{dy}{dx}=\frac{du}{dx}\cdot\frac{dy}{du}\]
Eksempel:
\[y=ln(2x+2)\]
\[u=2x+2\]
\[y=ln(u)\]
\[y'=\frac{dy}{dx}=\frac{du}{dx}\cdot\frac{dy}{du}=\frac{1}{u}\cdot2=\frac{2}{2x+2}\]
%%%%%%%%%%%%
\subsection*{Related rates - eksempel}
Vi har et rektangel:
Sidene x=10 og y=8.
\[\frac{dx}{dt}=2\]
\[\frac{dy}{dt}=-3\]
\[A=xy\]
Vi deriverer med produktregel
\[\frac{dA}{dt}=y\frac{dx}{dt}+x\frac{dy}{dt}\]
%%%%%%%%%%%%
\subsection*{Newton-Raphton metode}
En metode for å finne f(x)=0
\newline Tangentlinjen i punktet$(x_0,f(x_0))$er:
\[y=f(x_0)+f'(x_0)(x-x_0)\]
Punktet$(x_1,0)$ ligger på linjen slik at:
\[0=f(x_0)+f'(x_0)(x_1-x_0)\]
Loser med hensyn pa $x_1$
\[x_{n+1}=x_n-\frac{f(x_n)}{f'(x_n)}\]
Krav:$f'(x)\neq 0$
\newline Eksempel:
\newline Velger $x_0 = 2$
\[f(x)=x^2-2\]
\[f'(x)=2x\]
\[x_1=x_0-\frac{f(x)}{f'(x)}=2-\frac{2}{4}=1.5\]
\[x_2=x_1-\frac{f(x)}{f'(x)}=1.5-\frac{0.25}{3}=1.41667\]
\[x_3=x_2-\frac{f(x)}{f'(x)}=1.41667-\frac{0-00694}{2.8333}=1.4142\]
%%%%%%%%%%%%
\subsection*{Derivasjonsregler}
\[\frac{d}{dx}\sin^{-1}x=-\frac{1}{\sqrt{1-x^2}}\]
\[\frac{d}{dx}\sin^{-1}x=\frac{1}{\sqrt{1-x^2}}\]
\[\frac{d}{dx}\tan^{-1}x=\frac{1}{1+x^2}\]
\[\frac{d}{dx}\]


%%%%%%%%%%%% Partiell derivasjon %%%%%%%%%%%%
\newpage\section{Partiell derivasjon}


%%%%%%%%%%%% Integrasjon %%%%%%%%%%%%
\newpage\section{Integrasjon}
\underline{Potens}
\[\int x^{n} dx = \frac{1}{n+1}\cdot x^{n+1} + C, r\neq -1\]
\[\int \sqrt{x} dx = \frac{2}{3} \cdot x^{\frac{3}{2}} + C\]
\[\]
\[\]
%%%%%%%%%%%%
\subsection*{Delbrokoppspaltning}
$\int\frac{10}{t^2-t-6}dt=\int\frac{10}{(t-3)(t+2)}dt=\int\frac{A}{(t-3)}+\frac{B}{t+2}dt$
\newline Bestemmer A og B
\newline$10=(t+2)A+(t-3)B$
\newline$t=3\Rightarrow 10=(3+2)A+(3-3)B\Rightarrow A=2$
\newline$t=-2\Rightarrow 10=(-2+2)A+(-2-3)B\Rightarrow B=-2$
\newline Fullforer integrasjonen
\newline$\int\frac{2}{t-3}-\frac{2}{t+2}dt=\int\frac{2}{t-3}dt-\int\frac{2}{t-3}dt=2ln\mid t-3\mid-2ln\mid t+2\mid + C$
%%%%%%%%%%%%
\subsection*{Volum ved hjelp av integrasjon}
Volum av en gitt sylinder er:
\[V=\pi\cdot r^{2}\cdot h=\pi\cdot r^{2}\cdot\Delta x\]
Ved å summere sylindrene fra a til b og la hoyden$\Delta x$ ga mot 0 får vi
\newline volumet av omdreiningslegemet.
\newline Dette tilsvarer definisjonen pa det bestemte integralet fra a til b.
\newline Derfor er volumet av omdreningslegemet gitt ved:
\[V=\pi\int_{a}^{b}(f(x))^{2}dx\]
\subsection*{Noen integraler}
Ved delbroksoppspaltning
\newline$\int\frac{1}{b+x}dx=ln\mid b+x\mid+C$
\newline$\int\frac{1}{b-x}dx=-ln\mid b-x\mid+C$
\[\int\frac{1}{x}dx = ln[x]+C\]
\[\int\frac{1}{x}dx = ln[x]+C\]
\[\int\frac{1}{x}dx = ln[x]+C\]
\[\int\frac{1}{x}dx = ln[x]+C\]

%%%%%%%%%%%% Tallfolger og Rekker %%%%%%%%%%%%
\newpage\section{Tallfolger og Rekker}
{\bf Definisjon folge:}
\newline En folge er en uendelig samling av tall i en bestemt rekkefølge
\newline {\bf Definisjon rekke:}
\newline En rekke er en uendelig sum av tall
\newline Vi skriver dette som $\sum^{\infty}_{n=1} a_n = a_1 + a_2....$
\newline $\sum^{\infty}_{n=1}a_n = \lim_{N\to\infty}\sum^{N}_{n=1}a_n$
%%%%%%%%%%%%
\subsection*{Egenskaper til rekker}
\begin{itemize}
\item $a_1,a_2,....$ er ledd nr.n i folgen.
\item L er nedre grense for $a_n$ dersom $a_n \geq L$
\item L er ovre grense for $a_n$ dersom $a_n \leq L$
\item Rekka er positiv dersom $a_n > 0$ for n = 1,2,3,4....
\item Rekka er okende dersom $a_{n+1} > a_n$
\item Rekka er avtagende dersom $a_{n+1} < a_n$
\item Rekka er alternerende dersom annenhvert ledd er positivt og negativt.
\end{itemize}
%%%%%%%%%%%%
\subsection*{Konvergens av rekker}
\begin{itemize}
\item At en rekke konvergerer betyr at den har en sum.
\item Hvis vi ser pa leddene i en konvergent
rekke vil leddene bli mer lik en verdi jo senere leddet ligger i rekken.
\newline F.eks vil $a_{n}$ ligge naermere denne verdien enn det $a_{n-1}$ gjør.
\newline Vi sier at rekka konvergerer mot denne verdien.
\item En rekke enten konvergerer eller divergerer.
\end{itemize}
%%%%%%%%%%%%
\subsection*{Tips for rekkefolge}
\begin{itemize}
\item Er rekka geometrisk, p-rekke eller
\newline (opplagt) teleskoprekke?
\item Sjekk med n-te ledd test
\item Hvis alle ledd er positive:
\newline Prøv forholdstest($n!$ eller potense av n)
\newline Prøv integraltest(hvis funksjon som lett kan integreres, f.eks trigonometriske funksjoner og logaritmer)
\newline Prøv rottest.
\item Hvis ikke alle ledd er positive:
\newline Sjekk absolutt konvergens med en av testene over.
\newline Alternerende? Bruk Alternerende rekkers test.
\end{itemize}
%%%%%%%%%%%%
\subsection*{Teleskoprekker}
%%%%%%%%%%%%
\subsection*{Geometrisk rekke}
\[\sum_{n=1}^{} a \cdot k^{n-1} = a + ak + ak^{2}..\quad,k \neq 1\]
\newline Delsummen er gitt ved: $a_1 \cdot \frac{k^n -1}{k-1}$
\newline En uendelig geometrisk rekke konvergerer bare hvis $|k|<1$
\newline Da er summen, s $= \frac{a_1}{1-k} \quad$
%%%%%%%%%%%%
\subsection*{Potensrekker}
Hvert ledd er i tilegg avhengig av enda en variabel x.
\[\sum^{\infty}_{n=0} a_{n} \cdot (x-a)^{n}\ = a_0 + a_1 \cdot (x-a) + a_2 \cdot (x-a)^{2}\]
{\bf Konvergens av potensrekker}
\begin{itemize}
\item Konvergensradiusen er et tall\newline R$\in(0,\infty)$
\item De to grenseverdiene som gir $|x-a|$ ma sjekkes for konvergens.
\item Konvergensintervallet blir da \newline$(-R+a,R+a)$
\item $|x-a|<R\newline\Rightarrow$ innenfor konvergensintervallet.
\item $|x-a|>R\newline\Rightarrow$ utenfor konvergensintervallet.
\item Konvergerer rekken for x=a, sier vi at R=0.
\item Konvergerer rekken for alle x sier vi at R $=\infty$ 
\end{itemize}
%%%%%%%%%%%%
\subsection*{Funksjon - Rekke}
\begin{itemize}
\item$e^x =\sum^{\infty}_{n=0}\frac{x^n}{n!} = 1+x+\frac{x^2}{2!}+\frac{x^3}{3!}....$
\item$\sin x =\sum^{\infty}_{n=0}\frac{(-1)^n\cdot x^{2n+1}}{(2n + 1)!} = x - \frac{x^3}{3!} + \frac{x^5}{5!}....$ 
\item$\cos x =\sum^{\infty}_{n=0}\frac{(-1)^n\cdot x^{2n}}{(2n)!} = 1 - \frac{x^2}{2!} + \frac{x^4}{4!}....$
\end{itemize}
%%%%%%%%%%%%
\subsection*{n-te ledd test}
$\sum^{\infty}_{n=1}$konvergerer dersom $\lim_{n\to\infty}a_n=0$
\newline Dersom $\lim_{n\to\infty}a_n$ ikke eksisterer eller $\neq 0$
\newline$\Rightarrow$ rekken divergerer
%%%%%%%%%%%%
\subsection*{Forholdstesten}
\underline{$Dersom \lim_{n\to\infty} \frac{a_{n+1}}{a_n} = p \quad eksisterer$}
\begin{itemize}
\item $0\leq p \leq 1 \Rightarrow Rekken \quad konvergerer$
\item $p > 1 \Rightarrow Rekken \quad divergerer$
\item $p = 1 \Rightarrow ingen \quad konklusjon$
\end{itemize}


%%########## Vektorregning ############
\newpage\section{Vektorregning}
%%%%%%%%%%%%
\subsection*{Skalarprodukt/Dot product}
{\em Skalarproduktet $\vec a \cdot \vec b$ gir et reelt tall}
\[\vec p = [p_1,p_2,p_3], q = [q_1,q_2,q_3]\]
\[\vec p \cdot \vec q = |\vec p| \cdot |\vec q| \cdot cos \theta \quad, 0 \leq \theta \leq \pi\]
\underline{Koordinatformelen for skalarprodukt}
\[\vec p \cdot \vec q = \vec p_1 \cdot \vec q_1 + \vec p_2 \cdot \vec q_2 + \vec p_3 \cdot \vec q_3\]
%%%%%%%%%%%%
\subsection*{Vektorprodukt/Cross product}
{\em Vektorproduktet $\vec a \times \vec b$ gir en ny vektor som star vinkelrett pa $\vec a og \vec b$}
\[|\vec v \times \vec u | = |\vec v| \cdot |\vec u| \cdot sin \theta \quad, 0 \leq \theta \leq \pi\]
\[\vec p \times \vec q = -(\vec q \times \vec p)\]
\underline{Koordinatformelen for vektorprodukt}
\newline
{\em gitt vektorene:}
$
\vec a = [4,2,-3], \vec b = [2,-3,4]
\newline \vec a \times \vec b =
\begin{vmatrix}
\vec e_x&\vec e_y&\vec e_z\\
\vec 4&2&-3\\
\vec 2&-3&4\\
\end{vmatrix}
\newline
=\vec e_x
\begin{vmatrix}
\\2&-3
\\-3&4
\end{vmatrix}
-\vec e_y\begin{vmatrix}\\4&-3\\2&4\end{vmatrix}
+\vec e_z
\begin{vmatrix}
\\4&2
\\2&-3
\end{vmatrix}
$

%%%%%%%%%%%%
\subsection{Unit vektor}

%%%%%%%%%%%%
\subsection{Skalar projeksjon}

%%%%%%%%%%%%
\subsection{Vektor projeksjon}
\[\]
\end{document}